\documentclass[a4paper]{scrartcl}

\usepackage[ngerman]{babel}
\usepackage{amsmath,amssymb,amsthm,amsfonts,amsbsy,latexsym}

\usepackage[utf8]{inputenc}


\usepackage[T1]{fontenc}
\usepackage{enumerate,url}
\usepackage{graphicx}
\usepackage{bibgerm}
\usepackage[babel,german=guillemets]{csquotes}
%\usepackage{biblatex}
%\usepackage[style=numeric-comp]{biblatex}
\usepackage{listings}
\usepackage{color}
\usepackage[svgnames]{xcolor}

\usepackage{amsmath}
\usepackage{amssymb}
\usepackage{amstext}
\usepackage{amsfonts}
\usepackage{mathrsfs}
\usepackage{listings}		% Quelltext verwenden
\usepackage{color}

\usepackage{fancybox}  	% Box um Formel
 \usepackage{varwidth}
 
 \usepackage{lscape}

%
%   Hier beginnt das Hauptdokument
%

\begin{document}
\thispagestyle{empty}

%
%	Titelseite
%
\thispagestyle{empty}
\begin{center}
\Large{Hochschule für Technik und Wirtschaft Berlin (HTW)}\\
\end{center}
 
 
\begin{center}
\Large{Fachbereich 4 - Informatik, Kommunikation und Wirtschaft}
\end{center}
\begin{verbatim}


\end{verbatim}
\begin{center}
\textbf{\LARGE{Belegarbeit}}
\end{center}
\begin{verbatim}
 
 
\end{verbatim}
\begin{center}
\textbf{im Studiengang Angewandte Informatik}
\end{center}
\begin{verbatim}
\end{verbatim}
 
\begin{flushleft}
\begin{tabular}{lll}
\textbf{Fach:} & & Mathematik 3\\
& & \\
& & \\
\textbf{Thema:} & & Beleg 2 - Iteration\\
\textbf{Bschreibung:}& & Näherungsverfahren einer Gleichung \\
\textbf{}& & nach dem Tangentenverfahren von Isaac Newton \\
& & \\
& & \\
\textbf{eingereicht von:} & & Alexander Lüdke,\\
& & MatrNr.: 548965\\
& & \\
\textbf{eingereicht am:} & & 25.11.2015, Wintersemester 2015/16 \\
& & \\
& & \\
\textbf{Dozentin:} & & Dipl. Math. Petra Schumann
\end{tabular}
\end{flushleft}

\newpage

\thispagestyle{empty}

\tableofcontents
%\maketitle

\newpage
\setcounter{page}{3}
\section{Einleitung}
Diese Belegarbeit stellt die erste von zwei Semesterbeurteilungen  dar und befasst sich mit der Berechnung  des
Näherungsverfahren einer Gleichung nach dem Tangentenverfahren von Isaac Newton mittels GNU Octave. Dieses 
Dokument ist grob in drei Teile untergliedert.
\begin{enumerate}
	\item Einleitung und Aufgabenstellung
	\item Durchführung der Newton Iteration
	\item Struktogramme und Befehlsreferenzen
\end{enumerate}


\section{Belegaufgabe}
Für die Funktion $y= (x-3) \cdot (x+2)$ soll die positive Nullstelle bestimmt werden. Ermitteln Sie diese
nach der Newton-Iteration aus der Funktionsdeklaration. Der Startwert $x_{0}$ soll beliebig eingebbar
sein. Anhand der Lipschitzbedingung ist zu prüfen, ob es sich um einen optimalen Startwert handelt.
Das Prüfergebnis ist als Kommentar auszugeben. Es sollen alle Iterationsschritte protokolliert 
werden, der Abbruch kann nach der k'ten Iteration erfolgen, falls $| f(xk) |< (eps=0.0005)$. 
Geben Sie die Iteration grafisch aus.

\section{Manuelle \glqq{}händische\grqq{}  Berechnung}

% Ableitung bilden
	\subsection{Ableitungen bilden}
	\begin{tabular}{lll}
		\textbf{Eingangsfunktion:} & & $f(x) =  (x-3) \cdot (x+2) \Leftrightarrow x^2-x-6$ \\
		\textbf{erste Ableitung:} & & $f'(x) = 2x-1$\\
		\textbf{zweite Ableitung:} & & $f''(x) = 2$\\
	\end{tabular}
	
% Startwert beestimmen
	\subsection{Startwert bestimmen durch Prüfung der Lipschitzbedingung}
		
	% Lipschitzformel
		\fbox{
		\begin{tabular}{lll}
		\textbf{Lipschitzformel:} & & 
		$
		\left| \dfrac{f(x_{0}) \cdot f''(x_{0})}{f'(x_{0})^{2}} \right|< 1
		$
		\end{tabular}
		}
		
		\begin{tabular}{lll}
	% x0 Wert
		\\
		\textbf{$x_{0}$ festlegen:} & & $x_{0} = 2$ \\
		\textbf{$x_{0}$ einsetzen:} & & $f(2) = 2^2-2-6 = -4$ \\
		\textbf{} & & $f'(2) = 2 \cdot 2 -1$ = 3 \\
		\textbf{} & & $f''(2) = 2$ \\ \\
		
	% Lipschitzformel anwenden
		\textbf{Lipschitzformel anwenden:} & &
			$
			\left| \dfrac{f(2) \cdot f''(2) \vert}{\vert f'(2)^{2}}\right| = 
			\left| \dfrac{ -4 \cdot 2}{9}\right| =
			\left| \dfrac{-8}{9}\right| =$ \\ \\
			& &
			$
			= \left| -0,\overline{8}\right| \approx \vert -0,8889 \vert   < 1 
			$
			
		\end{tabular}
		
	% Newton Iteration	
	\subsection{Newton Iteration durchführen}
		
	% Newton Iteration Formel
		\fbox{
			\begin{tabular}{lll}
				\textbf{Newton-Iteration Formel:} & &
				$
				x_{k+1} = x_{k} - \dfrac{f(x_{k})}{f'(x_{k})}
				$
			\end{tabular}
			}

		\begin{tabular}{lll}
		\\
		\textbf{Abbruchbedingung:} & & $| f(x_{k}) |< (eps = 0.0005)$ \\ \\
		
	%  1. Iterationen
		\textbf{1. Iteration:} & &
		$
		x_{1} = x_{0} - \dfrac{f(x_{0})}{f'(x_{0})} = 2 - \dfrac{-4}{3} = 3,\overline{3}  \approx 3,3333
		$ \\ \\
		\textbf{} & &
		$
		f(3,3333) = 1,77758889 \approx 1,7776 \not< eps
		$ \\ \\
		
	%  2. Iterationen
		\textbf{2. Iteration:} & &
		$
		x_{2} = x_{1} - \dfrac{f(x_{1})}{f'(x_{1})} = 3,3333 - \dfrac{1,7776}{5,6666} \approx 3,0196
		$ \\ \\
		\textbf{} & &
		$
		f(3,0196) = 0,09838416  \approx 0,09838  \not< eps
		$ \\ \\
		
	%  3. Iterationen
		\textbf{3. Iteration:} & &
		$
		x_{3} = x_{2} - \dfrac{f(x_{2})}{f'(x_{2})} = 3,0196  - \dfrac{0,09838}{5,0392} \approx 3,00008
		$ \\ \\
		\textbf{} & &
		$
		f(3,00008) = 0,09838416  \approx 0,0004  < eps
		$ \\
		\end{tabular}

\section{Umsetzung in GNU Octave}
	\subsection{Eingabe der Funktion}
		Die Eingabe der Grundfunktion $(\textbf{1}\cdot x^2\textbf{-1}\cdot x^1\textbf{-6} \cdot x^0)$  wird in Form der Koeffizientendarstellung eingetragen. \\ \\
		\textbf{$>>$ grundFunktion = [1 -1 -6]; }
	
	\subsection{Startwert eingeben und Prüfung der Lipschitzbedingung}
		Zur Überprüfung des optimalen Startwertes, wird an dieser Stelle die Funktion \textit{lipschitzBedingung()} aufgerufen und die 
		zuvor erstellte Grundfunktion \textit{grundFunktion} als Parameter übergeben. Der Rückgabewert wird an die Variable \textit{lipschitzWert} übergeben, wenn die Bedingung
		\textit{Lipschitzwert < 1} erfüllt ist. \\ \\
		\textbf{$>>$ lipschitzWert = lipschitzBedingung(grundFunktion); }
			
	\subsection{Newton Iteration durchführen}
		Nachdem die Lipschitzbedingung erfüllt, d.h. der optimale Startwert durch den Benutzer eingegeben wurde, wird die Funktion \textit{newtonIteration()} aufgerufen. 
		Die Übergabewerte sind die  Grundfunktion \textit{grundFunktion}, die Abbruchbedingung bzw. Fehlertoleranz (eps) \textit{5e-4} und der Lipschitzwert \textit{lipschitzWert}. \\ \\
		\textbf{$>>$ newtonIteration(grundFunktion, 5e-4, lipschitzWert)}
		
	\subsection{Visualisierung}
	
	\subsubsection{textuelle Ausgabe}
		\lstset{language=}
			\lstinputlisting[emphstyle=\underbar, 
			breaklines=true,
			frame=tlrb]
			{./textAusgabe}
			
	\subsubsection{grafische Ausgabe}
		\begin{figure}[ht]
			\centering
			\includegraphics[scale=0.5]{plot.pdf}
		\caption{Grundfunktion und Tangenten}
		\end{figure}
	
	\section{Skript (.m-Datei)}
		\subsection{mainNewtonIteration.m}
			\lstset{language=}
			\lstinputlisting[emphstyle=\underbar, 
			breaklines=true, 
			frame=tlrb,
			caption={mainNewtonIteration.m}]
			{./mDaten/mainNewtonIteration.m}
			
		\subsection{lipschitzBedingung.m}
			\lstset{language=}
			\lstinputlisting[emphstyle=\underbar, 
			breaklines=true, 
			frame=tlrb,
			caption={lipschitzBedingung.m}]
			{./mDaten/lipschitzBedingung.m}
			
		\subsection{newtonIteration.m}
			\lstset{language=}
			\lstinputlisting[emphstyle=\underbar, 
			breaklines=true, 
			frame=tlrb,
			caption={newtonIteration.m}]
			{./mDaten/newtonIteration.m}
			
		\subsection{tangentenBerechnung.m}
			\lstset{language=}
			\lstinputlisting[emphstyle=\underbar, 
			breaklines=true, 
			frame=tlrb,
			caption={tangentenBerechnung.m}]
			{./mDaten/tangentenBerechnung.m}	
		
\newpage

\begin{landscape}
	\begin{figure}[ht]
	\section{Struktorgramme}
			\centering
			\includegraphics[scale=0.8]{./struktogramme/zusammenfassungStrukto.pdf}
		\caption{Struktogramme}
		\end{figure}	
\end{landscape}


\section{Befehlsreferenz}
		\begin{itemize}
			\item \textbf{lipschitzBedingung}    Funktion zur Berechnung des optimalen Startwertes.
			\item \textbf{newtonIteration()}    Funktion zur Durchführung der Newton Iteration, welche gleichzeitig die entsprechende Funktion
			grafisch ausgibt.
			\item \textbf{tangentenBerechnung()}  Funktion dient in erster Linie zur grafischen Ausgabe der Tangenten innerhalb der Newton Iteration. 	
			\item \textbf{clc} Löscht die bisherigen Ausgabe vom Befehlsfenster.
			\item \textbf{clear * all} Löst die Variableninitialisierung.
			\item \textbf{polyder()} Bildung von Ableitungen
			\item \textbf{polyval()} Berechnung Funktionswert
			\item \textbf{disp()} Textausgabe
			\item \textbf{printf()} Formatierte Textausgabe
			\item \textbf{plot()} Grafische Ausgabe
			\item \textbf{do-until} Wiederholung (Schleife)
			\item \textbf{if-else-endif} Alternation (Bedingung)
		\end{itemize}	
		
\end{document}