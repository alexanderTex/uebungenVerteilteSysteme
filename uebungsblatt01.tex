\documentclass[a4paper]{scrartcl}

\usepackage[ngerman]{babel}
\usepackage{amsmath,amssymb,amsthm,amsfonts,amsbsy,latexsym}

\usepackage[utf8]{inputenc}


\usepackage[T1]{fontenc}
\usepackage{enumerate,url}
\usepackage{graphicx}
\usepackage{bibgerm}
\usepackage[babel,german=guillemets]{csquotes}
\usepackage{listings}
\usepackage{color}
\usepackage[svgnames]{xcolor}

\usepackage{amsmath}
\usepackage{amssymb}
\usepackage{amstext}
\usepackage{amsfonts}
\usepackage{mathrsfs}
\usepackage{listings}		% Quelltext verwenden
\usepackage{color}

\usepackage{fancybox}  	% Box um Formel
\usepackage{varwidth}

\usepackage{lscape}

\usepackage{paralist}	% Aufzaehlunge wie a), b)
\usepackage{enumitem}


% /===============================================================================================
%
%		Dokumentanfang
%
% \===============================================================================================

\begin{document}
\thispagestyle{empty}

% /===============================================================================================
%
%		Deckblatt
%
% \===============================================================================================

\thispagestyle{empty}
\begin{center}
\Large{Hochschule für Technik und Wirtschaft Berlin (HTW)}\\
\end{center}


\begin{center}
\Large{Fachbereich 4 - Informatik, Kommunikation und Wirtschaft}
\end{center}
\begin{verbatim}


\end{verbatim}
\begin{center}
\textbf{\LARGE{Übungsaufgaben}}
\end{center}
\begin{verbatim}


\end{verbatim}
\begin{center}
\textbf{im Studiengang Angewandte Informatik}
\end{center}
\begin{verbatim}
\end{verbatim}

\begin{flushleft}
\begin{tabular}{lll}
\textbf{Fach:} & & Verteilte Systeme\\
& & \\
& & \\
\textbf{Thema:} & & Übungsblatt 1\\
& & \\
& & \\
\textbf{eingereicht von:} & & Alexander Lüdke (548965)\\
& & Sascha Bussian (549087)\\
& & \\
\textbf{eingereicht am:} & & Sommersemester 2016 \\
& & \\
& & \\
\textbf{Dozenten:} & & Prof. Dr. Albrecht Fortenbacher\\
& & Tobias Gruber \\
\end{tabular}
\end{flushleft}

\newpage

% /===============================================================================================
%
%		Inhaltsverzeichnis
%
% \===============================================================================================

\thispagestyle{empty}

\tableofcontents
%\maketitle

\newpage

% /===============================================================================================
%
%		Hauptteil
%
% \===============================================================================================

\setcounter{page}{3}
\section{Übung 01}



\subsection{Definition verteilter Systeme}
Das System a), eine dezentral organisierte Büroumgebung auf einem Workstation-Netz, und d) ein Grid-System sind nach der Definition von Schill und Springer [2012,4] und Günther Bengel [2014,12ff] verteilte Systeme.

\subsection{Ressourcen}
	\subsubsection{Hardware-Ressourcen}
		\begin{itemize}
			\item Festplatten als Massenspeicher
			\item Bandlaufwerke zur Datensicherung
			\item Drucker
			\item CPU
			\item GPU (Rechenintensive Berechnungen)
		\end{itemize}
	\subsubsection{Daten- und Software-Ressourcen (Verteilte Anwendungen)}
		\begin{itemize}
			\item Datenbanksystem
			\item Cloud-Printing
			\item Applikation
			\item Gemeinsam genutzte Log-Files
		\end{itemize}
	\subsubsection{Anwendungsbeispiele aus beiden Ressourcen-Typen}
		\begin{itemize}
			\item Einfache verteilte Anwendungen (www, E-Mail)
			\item Komplexe Anwendungen (Grid Computing, Bürokommunikation, Wettersimulation)
			\item Eingebettete verteilte Systeme (Automobilindustrie, Hausautomation)
			\item Mobile verteilte Systeme (portable Geräte, Mobilfunk)
		\end{itemize}
		[Irmscher, Klaus (2013), Scriptum zur Lehrveranstaltung Verteilte Systeme. Online: \url{https://www.daf.tu-berlin.de/fileadmin/fg75/PDF/Zitieren.pdf}; Seite 7; Stand: 18.04.2016]



\subsection{Sicherheit}
%Fußnote - Quelle: Schill Springer [2012:133ff]
Der Bereich Sicherheit in verteilten Systemen untergliedert sich in vier Kategorien und kann auf das genannte Beispiel wie folgt interpretiert werden.
	\begin{itemize}

		\item \textbf{Verhindern} \\
		Der Dienst wird mit Anfragen überflutet, was dazu führt, dass die Teilnehmer nicht mehr in der Lage sind Nachrichten auszutauschen. (DOS)

		\item \textbf{Erlangen} \\
		Das Abfangen der Daten durch Dritte führt zu einer Beeinträchtigung der Vertraulichkeit, sowie den Zugriffsschutz der Ressourcen. (Man-in-the-middle)

		\item \textbf{Modifizieren und Fälschen} \\
		Das Verändern der Daten durch Dritte würde zu Integritätsverlust führen, darüber hinaus ist es möglich die Betroffenen in ihrem Handeln zu manipulieren.

	\end{itemize}
Um diesen Sicherheitsproblemen entgegenzuwirken wurden Sicherheitsmechanismen bereitgestellt, die in Folgende Schutzziele unterteilt sind:
	\begin{itemize}

		\item \textbf{Vertraulichkeit} \\
		Die Inhalte der Kommunikationsnachrichten müssen vor unberechtigten Teilnehmern geschützt werden.
        Z. B. durch Verschlüsselung der Kommunikationswege.

		\item \textbf{Integrität} \\
		Die übertragenen Daten dürfen nicht unberechtigt verändert worden sein bzw. eventuelle Änderungen müssen erkannt werden.
    	Z. B. durch Checksums.

		\item \textbf{Verfügbarkeit} \\
		Die Bereitstellung der Dienste, Informationen und Kommunikationsverbindungen muss zum Zeitpunkt der Kommunikation bereitstehen.

		\item \textbf{Authentizität} \\
		Es stellt die Überprüfbarkeit bzw. Echtheit der Nachricht und des Absenders sicher.

		\item \textbf{Zugriffsschutz} \\
		Die Verhinderung von Zugriff auf Ressourcen durch unberechtigten Zugriff.

		\item \textbf{Zurechenbarkeit} \\
		Ein eindeutige Zuordnung von Interaktionen zu eine Instanz bzw. Person.

		\item \textbf{Anonymität} \\
		Diese steht der Authentizität gegenüber und soll den Benutzern die Möglichkeit bieten, eine Kommunikation ohne Preisgabe der persönlichen Daten aufzubauen wobei man hier den Einsatz von Pseudonymen vorsieht.
	\end{itemize}
Um die Schutzziele Vertraulichkeit, Integrität und Authentizität zu realisieren, verwendet man  unterschiedliche Verschlüsselungsverfahren (Symmetrische- und Asymmetrische Kryptoverfahren). Diese erlauben die gesendeten Nachrichten zu ver- und entschlüsseln, sowie ihnen eine digitale Signatur anzuhängen. Mittels Autorisierung in Form von Zugriffsmatrizen, den daraus resultierenden Zugriffskontrolllisten (Access Control List, ACL) und die Einteilung in Gruppen und Rollen ist es möglich die Ziele des Zugrifftsschutz und der Zurechenbarkeit zu realisieren. Zusätzlich sollten die Systeme, die zum Nachrichtenaustausch eingesetzt werden mittels Firewall geschützt werden.



\subsection{3-tier-Architekturen}
Diese Architektur ist in drei Schichten (tiers) unterteilt, der Benutzerschnittstelle (presentation tier), der Anwendungs- Logikschicht (logic tier) und Datenschicht (data tier). Diese bauen von unten nach oben aufeinander auf, wobei die obere Schicht immer von den Schichten darunter abhängig ist. D.h., wenn die untere Schicht verändert wird, müssen auch die darüber liegenden Schichten angepasst werden. Ein Beispiel ist das Aufrufen von Websites. Hierbei wird eine Anfrage für eine Website an einen Webserver als Einstiegspunkt gewählt. Dieser leite die Anfrage an den Anwendungsserver weiter, auf dem die eigentliche Verarbeitung stattfindet, der wiederum mit einem Datenbankserver zusammenarbeitet. Das Ergebnis der Anfrage vom Webserver wird in Anschluss auf umgekehrten Weg wieder zurück an ihn gesendet (Datenbankserver -> Anwendungsserver -> Webserver).

Quelle: http://www.mrknowing.com/2013/11/08/wie-funktioniert-die-3-schichten-architektur/; Stand: 20.04.2016 \\ Tanenbaum \& van Steen [2008:61f]

\subsection{Verteilung}
	\begin{itemize}
		\item horizontale Verteilung \\
		Hierbei können ein Client oder ein Server physisch in logisch gleichwertige Teile gegliedert werden. Wobei jeder dieser Teile mit einem eigenen Anteil der vollständigen Datenmenge arbeitet, was dazu führt, dass die Gesamtlast verteilt wird.

		\item vertikale Verteilung \\
		In dieser Architektur entsprechen die Tiers direkt der logischen Anordnung der Anwendungen. D.h. die Funktionen sind logisch und physisch über mehrere Computer verteilt, wobei jeder Rechner auf eine bestimmte Gruppe von Funktionen zugeschnitten ist.
	\end{itemize}

\subsection{Uhrensynchronisierung}

	% Aufzählung mit kleinen Buchstaben und Klammer -> a), b), ...
	\renewcommand{\labelenumi}{\alph{enumi})}
	\begin{enumerate}
		\item Im lokalen Netzwerk behält jeder Rechner seine eigene lokale Zeit im Auge. Eine Möglichkeit trotz, relativer Ungenauigkeit, Daten auszutauschen, die einen Bezug zu Zeit haben, sind die logischen Uhren von Lamport. Logischen Uhren müssen nicht zwangsläufig mit der aktuellen (physischen) Zeit übereinstimmen. Hierbei geht es lediglich um die Ausstellung von Zeitstempeln die in Relation zu einander gesetzt werden(a geht b voraus [a -> b]). Ein Beispiel hierfür sind Makefiles.

Eine andere Möglichkeit ist, dass man innerhalb eines Netzwerkes einen Rechner bestimmt und dessen Zeit als Referenzzeit nimmt, dabei ist es egal, ob die Zeit nicht der echten Zeit entspricht. Einzig allein die Einigung einer gemeinsam genutzten Zeit zählt.

		\item Die Computer, die in der Lage sind auf das Internet zuzugreifen, lösen ihr Synchronisationsproblem durch einen Zeitserver, der die exakte Zeit bereitstellt, da er mit einer exakten Uhr ausgestattet ist. Das Problem hierbei ist die Nachrichtenverzögerung, nachdem ein Client eine Zeitanfrage stellt. Hierbei kommt das Network Time Protocol (NTP) zum Einsatz, welches durch Berechnung die zuverlässigste Schätzung der Abweichung ermittelt, was zu einer weltweiten Genauigkeit von 1 bis 50 ms führt.
	\end{enumerate}

\end{document}
